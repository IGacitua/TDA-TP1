\section{Consideraciones}

Este documento está pensado para que sirva de guía, e incluso template a la hora de realizar un informe para la materia (por no decir, cualquier documento formal en la facultad). La idea es utilizar \LaTeX, pura y exclusivamente para que tenga un formato acorde. Es decir, no hecho en un simple procesador de texto (como pudiese ser Word). 

Aquí sólo se mostraran algunas de las cosas que se pueden hacer con \LaTeX. Si se quiere hacer algo específico, o de otra forma, abunda la documentación (y artículos de StackOverflow) al respecto. 

Definimos ciertos lineamientos técnicos a considerar para los informes: 
\begin{itemize}
    \item En el informe debe estar incluido el código más importante (es decir, los algoritmos, no un main) cuando corresponda. 
    \item Debe estar explicado cómo se hacen los sets de prueba. Inclusive, si notan alguna falencia con esta generación que pueda afectar de alguna forma el análisis del resultado, se espera que lo enuncien (hay trabajos donde esto no sucederá, pero hay trabajos donde sí, especialmente cuando vemos aproximaciones). 
    \item Los gráficos deben estar bien generados: deben tener título, los ejes deben tener tanto qué refieren como qué unidad de medida usan (en caso de tenerla). En caso de tener gráficos superpuestos, debe quedar claro qué es cada gráfico. 
    \item \textbf{Importante}: en caso de tener que realizar una reentrega (y por ende, tener que cambiar cosas del informe), salvo que haya que rehacer una enorme cantidad del mismo, no modificar el informe anterior sino agregar un anexo al final indicando las correcciones realizadas en la nueva entrega, para agilizar las correcciones. 
\end{itemize}

A su vez, dejamos algunas breves líneas sobre cuestiones de redacción: 
\begin{itemize}
    \item Evitar errores ortográficos y gramaticales. 
    \item El lenguaje debe ser técnico. No tiene que por eso ser un texto aburrido, pero evitar frases como "\textit{la cosa que más me complicó fue...}", "\textit{creo que...}". 
    \item Usar plural de modestia, incluso si el trabajo fuera realizado por una única persona.
\end{itemize}